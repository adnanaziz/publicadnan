\documentclass[fullpage,twocolumn]{article}
\usepackage{cancel}
\usepackage{color}
\usepackage{fullpage}

\newcommand{\vp}{{\vspace{0.1in}}}

\setlength{\parindent}{0cm}

%\title{Laila's Math PSIA Crib Sheet}
%author{Adnan Aziz}
%date{}
\begin{document}
%maketitle

\begin{center}
{\Large \bf Math Facts} \\
{\bf By Adnan Aziz \& Laila Aziz}

\end{center}


This document consists of mathematics notes developed by the authors for
tests.



\section{Gaussian pairing}
{\color{blue}
\begin{eqnarray*}
1 + 2 + 3 + 4 + 5 + 6 = \\
1 + 6 + 2 + 5 + 3 + 4 = \\
7 + 7 + 7 = 7 \times 3 & = & 21
\end{eqnarray*}
}

This works in many places:
\begin{eqnarray*}
3 + 6 + 9 + 12 + 15 + 18 + 21 = \\
3 + 21 + 6 + 18 + 9 + 15 + 12 = \\
24 + 24 + 24 + 12 = \\
24\times + 12 = 72 + 12 & = & 84 \\
\end{eqnarray*}

Formula: The sum of the numbers from 1 to $N$ is $\frac{N \times {N+1}}{2}$.

Example: if you have to add numbers from 10 to 20, it will
be all the numbers from 1 to 20 minus the numbers from 1 to 9.
The first sum is $\frac{20 \times 21}{2} = 210$ and the second
sum is $\frac{9 \times 10 }{2} = 45$. So the answer
is $210 - 45 = 165$.

Example: if you have to add all the even numbers from 2 to 40,
it's the same as 2 times all the numbers from 1 to 20.
So the answer is $2 \times \frac{20\times 21}{2} = 20\times 21 = 420$.

Formula: The sum of all the even numbers from $2$ to $2\times M$ is $\frac{M}{M+1}$.

Example: The sum of all the even numbers from $2$ to $100 = 2 \time 50$ is
$50 \times 51 =  2550$.

Formula: The sum of all the odd numbers from $1$ to $2 \times M  - 1$ is $M^2$.

Example: The sum of all the odd numbers from $1$ to $15 = 2 \times 8 - 1 $ is
$8^2 = 64$.




\section{Difference of squares}
${4^2 - 2^2 = (4 + 2) \times (4 - 2) =  6 \times 2 = 12  }$

$25^2 - 18^2 = (25+18)\times (25 - 18)$ \\
$ = 43\times 7 = 301$

$\mbox{Special case:}\;$\\
$10^2 - 9^2 = (10+9) \times (10-1) = 19$

$\mbox{Another special case:}\;$\\
$10^2 - 1 = 10^2 - 1^2 = (10+1) \times (10-1) = 11 \times 9 = 99$

$\mbox{Key fact:}\;   (a+1) \times (a-1) \;$ \\
$\mbox{is always one less than} \; a^2$

\section{Percentages}

\[
\mbox{Key formula:} \frac{\mbox{is}}{\mbox{of}} = \frac{\%}{100}
\]

\subsection{12 is what percent of 48?}

${\frac{12}{48} = \frac{\%}{100}} $\\
${\mbox{multiply both sides by 100}} $\\
${\frac{100 \times 12 }{48} = \frac{100 \times \cancel{12}}{4 \times \cancel{12}} = \frac{100}{4} = 25 = \frac{100 \times \% }{100} = \%}$
So the answer is 25

\subsection{What is 15\% of 60?}

${\frac{\mbox{is}}{60} = \frac{15}{100} }$ \\
${\mbox{multiply both sides by 60} }$ \\
${\frac{60 \times \mbox{is} }{60 } = \frac{ 15  \times 60}{60}}$ \\
${\frac{\cancel{60} \times \mbox{is} }{\cancel{60} } = \frac{ 15  \times \cancel{60}}{\cancel{60}}}$ \\
 ${\mbox{is} = 15} $ \\
So the answer is 15

\subsection{40 is 20\% of what?}

${\frac{40}{\mbox{of}} = \frac{20}{100}}$  \\
${\mbox{flip both sides} }$ \\
${\frac{\mbox{of}}{40} = \frac{100}{20}}$  \\
${\mbox{multiply both sides by 40}} $ \\
${\frac{\mbox{of} \times \cancel{40}}{\cancel{40}} = \frac{100 \times 40}{20} = \frac{100 \times 2 \times \cancel{20}}{\cancel{20}}} $ \\
${\mbox{of} = 100 \times 2 = 200} $


\section{Subtracting fractions}
${\frac{2}{3} - \frac{3}{10}}$ \\
make the bottom parts the same by multiplying top and bottom \\
${= \frac{2 \times 10}{ 3 \times 10} - \frac{3 \times 3}{10 \times 3}}$ \\
${ = \frac{20}{30} - \frac{9}{30}}$ \\
${ = \frac{20 - 9 }{30 }} $\\
${ = \frac{11}{30}}$

\section{Dividing fractions}

$\frac{3}{4} \div \frac{5}{7} = \frac{3}{4} \times \frac{7}{5} = \frac{3 \times 7 }{4 \times 5}$

Be very careful when you have multiple fractions:

$\frac{3}{2} \div \frac{3}{4} \div \frac{3}{5}$ is $(\frac{3}{2} \div \frac{3}{4})\div \frac{3}{5}$
and {\bf not} $(\frac{3}{2} \div \frac{3}{4})\div \frac{3}{5}$

So $\frac{3}{2} \div \frac{3}{4} \div \frac{3}{5} = (\frac{3}{2} \times \frac{4}{3}) \times \frac{5}{3}$, which simplifies to $\frac{10}{3} = 3 \frac{1}{3}$.

Be smart about cancelling---usually, you should look at factors and try and cancel.

$\frac{28 \times 42}{49}$: instead of multiplying $28$ and $42$, write it as
$\frac{(7 \times 2 \times 2)\times (2 \times 3 \times 7)}{7 \times 7}$. Now you can cancel
both the $7$s in the denominator, and get the answer is $2 \times 2 \times 2 \times 3 = 24$.

\section{Order of operations}

Multiplication and division first. Then addition and subtraction.

So $2 + 3 \times 4 - 9 \div 3 = 2 + 12 - 3 = 9$

Regardless, parens are done first: $(2 + 3 ) \times 5 + 1 = 5 \times 5 + 1 = 26$.

\section{Decimals}

\subsection{Adding}

$1.23 + 4.56 = 5.79$. $1.9 + 2.7 = 4.6$. $1.234 + 0.999 = 2.233$.

\subsection{Subtracting}

$4.56 - 1.23 = 3.33$. $3.14 - 1.25 = 1.89$ $3.1 - 0.999 = 2.1001$.

\subsection{Multiplying}

$0.12 \times 0.34 = 0.0408$ (Multiply without the decimal, than bring it back.
You may need to add 0s in front.

\subsection{Dividing}

$\frac{12}{0.3} = 40$.

Approach 1: it's $\frac{12}{\frac{3}{10}} = \frac{12}{1}\times\frac{10}{3} = \frac{120}{3} = 40$.

Approach 2: first ignore the decimal point: $\frac{12}{3} = 4$. Now we need
to move left for numerator (0) and right for denominator (1), gives us $40$.

$\frac{12.3}{0.03}$ using Approach~2 is $\frac{123}{3} = 41$. Now we move one left
for numerator, and then two right for denominator, i.e., overall one right $ = 410$.

$4.41 \div 0.7$: ignore the decimals, get $441\div 7 = (420 + 21) \div 7 = 60 + 3 = 63$.
Now we need to move the decimal 2 left ($4.41$) and 1 right $0.7$, i.e, 1 left,
so the answer is $6.3$.

\section{Roman Numbers}

$DCCV = 500 + 200 + 5$. $MMXI = 2000 + 10  + 1$. $LIII = 53$ $IL = 49$. $MCMXC = 1990$



\section{Perfect squares}
{\tiny
\begin{tabular}{|c|c|c|c|c|c|c|c|c|c|} \hline
1 & 2 & 3 & 4 & 5 & 6 & 7 & 8 & 9 & 10 \\
1 & 4 & 9 & 16 & 25 & 36 & 49 & 64 & 81 & 100 \\ \hline
11 & 12 & 13 & 14 & 15 & 16 & 17 & 18 & 19 & 20 \\
121 & 144 & 169 & 196 & 225 & 256 & 289 & 324 & 361 & 400 \\ \hline
\end{tabular}
}

Also remember $25^2 = 625$.

\section{Primes}
2, 3, 5, 7, 11, 13, 17, 19, 23, 29, 31, 37, 41, 43, 47, 53, 59, 61, 67, 71, 73, 79, 83, 91, 97, 101    103, 107, 109, 113, 127, 131, 137, 139, 149, 151, 157, 163, 167, 173, 179, 181, 191, 193, 197, 199

\section{Powers}
{\tiny
\begin{tabular}{|c|c|c|c|c|c|} \hline
$2$ & $2^2 = 4$ & $2^3 = 8$ & $2^4 = 16$ & $2^5 = 32$  & \\
$2^6 = 64$ & $2^7 = 128$ & $2^8 = 256$ & $2^9 = 512$ & $2^{10} = 1024$  & \\
$3$ & $3^2 = 9$ & $3^3 = 27$ & $3^4 = 81$ & $3^5  = 243$ & $3^6 = 729$  \\
$5$ & $5^2 = 25$ & $5^3 = 125$ & $5^4 = 625$  & & \\
$6$ & $6^2 = 36$ & $6^3 = 216$ &     &         &  \\
$7$ & $7^2 = 49$ & $7^3 = 343$ &     &         & \\ \hline
\end{tabular}
}

\section{Magic numbers}
$3^4 = 9^2 = 81$, $9^3 = 729 = 81 \times 9$

$4^4 = 8^3 = 256$.

$24 \times 3 = 72, 24 \times 4 = 96, 24  \times 5 = 120, 24 \times 6 = 144$

\section{Divisibility rules}

A number is divisible by 2 exactly when last digit is one of $0,2,4,6,8$.

$1214, 8, 24, 100000$: yes. $123, 41, 471, 88889$ no.

A number is divisible by 3 exactly when the sum of its digits is divisible by
$3$, and the remainder is the remainder of the sum.

Instead of adding all the digits, you can ``cast out'' the 3s as you add them.

$123457$: $1$, $1+2=0 \; \mbox{since we cast out the 3}$, $0+3 = 0$,
$0+4 = 1$, $1+5 = 0$, $0+7=1$, so the remainder is 1.

Same rule for 9---``cast out'' the 9s.

$123457$: $1$, $1+2=3$, $3+4 = 7$,
$7+5 = 3 \; \mbox{since we cast out the 9}$,
$3+7=1$, so the remainder is 1.

A number is divisible by 5 exactly when the last digit is one of $0$ or $5$.

A number is divisible by 11 exactly when the sum of the odd digits minus the sum
of the even digits is divisible by 11.

For example: $34871903$. Even digits are $3+8+1+0=12$; odd
digits are $4+7+9+3=23$. Since $23-12=11$ we have $34871903$ is divisible by 11.


\section{Good fractions}

$\frac{1}{4} = 0.25$, $\frac{1}{8} = 0.125$.

From this, you can see for example that $\frac{3}{8} = 0.375$.

Think of these fractions in terms of quarters, pennies, and dollars.

More common fractions:
$\frac{24}{72} = \frac{1}{3}$, $\frac{24}{108} = \frac{2}{9}$, $\frac{36}{144} = \frac{1}{4}$

Common percentages: 10\% of 120 = 12; 50\% of 120 = $\frac{120}{2}$;
40\% of 80 = $80 \times \frac{2}{5}$.

$750 =  375 \times 2$

To get $\frac{22}{25}$ as a decimal convert the
denomintor to $100$: $\frac{4\times 22}{4\times 25} = \frac{88}{100} = 0.88$.

To get $\frac{203}{50}$ as a decimal convert the
denominator to $100$: $\frac{2 \times 203}{2 \times 50} = \frac{406}{100} = 4.06$.

$\frac{7}{40}$ as a decimal = $\frac{7 \times 2.5 }{40 \times 2.5} = \frac{17.5}{100} = 0.175$.

$48$ minutes is $\frac{48}{60}$ as a fraction of an hour, $\frac{12\times 4 }{12 \times 5} = \frac{4}{5}$. Same for $12, 24, 36, 72$ minutes.

$45$ minutes is $\frac{45}{60}$ as a fraction of an hours, $\frac{15 \times 3}{15 \times 4} = \frac{3}{4}$. Same for $15, 75$ minutes.


$\frac{?}{5} = \frac{3}{4}$, how to get $?$.

Multiply out the $5$: $5 \times \frac{?}{5} =  5 \times \frac{3}{4}$,
from which $? = \frac{15}{4}$.


\section{Multiplying decimals}

We want to compute $1.6 \times 0.4 \times 1.4$

First, ignore the decimals:
$16 \times 4 \times 14 = 64 \times 14 = 64 \times (10 + 4) = 640 + 64 \times 4 = 640 + 256 = 896$

Now check how many places there are after the decimal there are: 1 from 1.6, 1 from 0.4, and 1 from 1.4,
so the answer is {\bf 0.896}

\section{Means and medians}

Mean: add up all the numbers and divide by the total number of numbers
\begin{itemize}
\item Mean of $2,3,5,14$ is equal to $\frac{2+3+5+14}{4} = \frac{24}{4} = 6$
\end{itemize}

What number should you add to $9,10,11$ to get a mean of $11$?

The mean of $9,10,11$ is $\frac{9+10+11}{3} = 10$. To get a mean of $11$, you might think of adding the number
$12$, but that won't work since instead you'll get $\frac{9+10+11+12}{4} = \frac{32}{4} = 10\frac{1}{2}$. You need more than $12$ to counter the fact that there are $3$ numbers
already. Specifically you need $11 + 3\times(11-10)  = 14$.

Median: sort the numbers and pick the middle one (if the number of numbers is odd);
otherwise take the two middle ones, add them, and divide by two.
\begin{itemize}
\item Odd case: median of $2,4,12,5,6$ is median of $2,4,5,6,12$, which is $5$
\item Even case: median of $1,9,12,5$ is median of $1,5,9,12$, which is $\frac{5+9}{2}=7$
\item What number can we add to $3,4,7,9$ to make the median $7$?
\begin{itemize}
\item Answer: any number greater than or equal 7, e.g., $8, 9, 100,\ldots$
\end{itemize}
\end{itemize}

\section{Probability}

Question: if there are 21 red balls, 42 blue balls, and 14 green balls
in a bag, and you choose one at random, what is
the probability its color is red?
\vp
Answer: $\frac{21}{21+42+14} = \frac{7\times3}{7\times 3 + 7 \times 6 + 7 \times 2} = \frac{3}{3+6+2} = \frac{3}{11}$


Question: if there are 2 red balls and 3 blue balls in Bag 1, and 1 red ball and 1 blue balls in Bag 2, and
you pick one ball from each bag randomly, what is the probability that they are both red?

\vp
Answer: The probability that ball from first bag is red is $\frac{2}{5}$; the probability that
the ball from the second bad is red is $\frac{1}{2}$. So the probability that both are red
is $\frac{2}{5}\times \frac{1}{2} = \frac{1}{5}$.

\section{Combinatorics}

Question: How many 4 digit numbers can you write using 2, 4, 5, 7 exactly once?

\vp
Answer: $4 \times 3 \times 2 \times 1 = 24$.

Question: How many 4 digit numbers can you write using 2, 4, 5, 7 exactly once that are bigger than 3000?

\vp
Answer: there are 24 total numbers.

The 4 digit numbers starting with 2 cannot be bigger than 3000.

Any 4 digit number starting with 4, 5, or 7, is bigger than 3000.

There are 24 total numbers and 6 start with 2 (think about this), so $24-6=18$ is
the number of 4 digit numbers using 2, 4, 5, 7 that are bigger than 3000.

Question: How many 3 digit numbers can you write using 1, 2, 3, 4 if you are
allowed to use each digit as many times as you want?

\vp
Answer: It's $4 \times 4 \times 4 = 64$.

Question: How many 4 letter words can you write using {\em a, b, b, c} exactly once?

\vp
Answer: If the letters were all different, there would be $4 \times 3 \times 2
\times 1 = 24$ possibilities.

Here's the 24 possibilities, assuming the two {\em b}'s are different (call
them $b_1$ and $b_2$):

\begin{tabular}{llll}
$a b_1 b_2 c $  & $ b_1 a b_2 c $ & $b_2 a b_1 c  $ & $c a b_1 b_2 $  \\
$a b_1 c b_2 $  & $ b_1 a c b_2 $ & $b_2 a c b_1  $ & $c a b_2 b_1 $  \\
$a b_2 b_1 c $  & $ b_1 b_2 a c $ & $b_2 b_1 a c  $ & $c b_1 a b_2 $  \\
$a b_2 c b_1 $  & $ b_1 b_2 c a $ & $b_2 b_1 c a  $ & $c b_1 b_2 a $  \\
$a c b_1 b_2 $  & $ b_1 c a b_2 $ & $b_2 c a b_1  $ & $c b_2 a b_1 $  \\
$a c b_2 b_1 $  & $ b_1 c b_2 a $ & $b_2 c b_1 a  $ & $c b_2 b_1 a $  \\
\end{tabular}

However, because the 2 {\em b}'s are really the same, in each
column, half the words are the same as the other half. So we need
to divide by 2, so the answer is $\frac{24}{2} = 12$.

\section{Calculation tricks}

\subsection{Look for close ``nice'' numbers}
$499 + 496 = (500 - 1) + (500 - 4 ) = 500 + 500 - 1 - 4 = 1000 - 4 - 996$\\

$3 \times 2.49 + 2 \times 1.99 =$ \\
$ 3 \times 2.5 + 2 \times 2 - 0.03 - 0.02 = 11.5 - 0.05 = 11.45$
(Think of this as a problem with cents and dollars, treat each part separately.)

\subsection{Look out for ``operator precedence''}

Do the squaring first, then multiply, then add.

So if you want to calculate $3 \times 2^2 - 1$,
the answer is $3\times  4 -  1 = 12 - 1 = 11$.

\subsection{Cancel in fractions}

In $\frac{3 \times 4 \times 5 \times 6}{7 \times 6 \times 5}$ you can cancel
the 5 and the 6, even though they are far apart: $\frac{3 \times 4 \times \cancel{5}
\times \cancel{6}}{ 7 \times  \cancel{6} \times \cancel{5}} = \frac{3 \times 4}{7} = \frac{12}{7}$.

\subsection{Multiply fractions to find unknowns}

$\frac{?}{5} = \frac{3}{4} \Rightarrow \; \mbox{multiply by } ; 5$\\
$\frac{?}{\cancel{5}} \times \cancel{5} = \frac{3 \times 5}{4}$ so  $? = \frac{15}{4} = 3\frac{3}{4}$.

\subsection{Use distribution}

$\frac{(140 - 14)}{14} = \frac{140}{14} - \frac{14}{14} = 10 - 1 = 9$.

$9^3 - 9^2 = 9 \times 9 \times 9 - 9 \times 9 = 9 \times 9 \times (9 - 1)$.

So if you want to know what $9^3 - 9^2$ divided by $9^2$ is, it's
just $(9-1) = 8$. Similarly, $9^3 - 9^2$ divided by $9$ is $\frac{9 \times 9 \times (9 - 1)}{9} = 9 \times 8 = 72$

\section{Geometry}

\subsection{Triangle area}
Fact: the triangle inside a rectangle always has half the area
of the rectangle.


\vspace{3in}

\subsection{Angles}


Angles are acute (less than $90$ degrees), right (equal to 90 degrees) and obtuse (more than 90 degrees).

\subsection{Long side of a right angled triangle}

Fact: for a right triangle, the square of the long side is the sum of the square of
the short sides. So a right triangle with short sides $3$ and $4$ has it's long
side length $ = \sqrt(3 \times 3 + 4 \times 4) = \sqrt(9 + 16) = \sqrt(25) = 5$.j

If you double the lengths of the short sides, the long side doubles too, same
for tripling, or any multiplication. For example,
in a right triangle if the shorter sides are $6$ and $8$, then the long side
is $10$ (since $6 = 3 \times 2$ and $8 = 4 \times 2$).

Memorize: if the two short sides of a right triangle are $5$ and $12$ the third is
$13$.

\subsection{Sum of angles}
Sum of angles of triangle = 180 degrees

Sum of angles in a polygon center is 360 degrees

\subsection{Numbers of sides}

tri, quad, pent, hex, oct, non, decagon

\subsection{Perimeter and area}

Perimeter of square/triangle/rectangle/any polygon is lenght of its sides.

The number of posts for a rectangle of perimeter
20 is 10 if the posts are spaced 2 apart.

Area of rectangle is length times width.

Area of rhombus is half of diagonal times diagonal.

Example: Rhombus has short diagonal length = 10 and long diagonal length = 20,
its area is $\frac{10 \times 20}{2} = 100$.

Area of circle is $\pi \times r \times r$, where $r$ is the radius (which is half the diamater).

Volume of a cube of side 3 is $3 \times 3 \times 3$.

Volume of cylinder is area of base (which is a circle) times the
height of cylinder.

\subsection{Transformations}

Rotation, translation, reflection, expansion (same as dilation), contraction

% \vspace{3in}

\section{Word problems}

Question: A car travels at 50 miles per hour. How far does it
travel in $2\frac{1}{5}$ hours?

\vp
Answer: first write the time as a fraction, $\frac{11}{5}$ hours.
In 1 hour, the car travels 50 miles $\Rightarrow$ it travels $50 \times \frac{11}{5} = 110$ miles in
$2\frac{1}{5}$ hours.

Question: A car travels at 90 miles per hour. How long does it take to travel 120 miles?

\vp
Answer: It takes $\frac{1}{90}$ hours to travel 1 mile, so it takes $120\times \frac{1}{90} = \frac{120}{90} = \frac{4}{3} = 1 \frac{1}{3}$ hours to travel 120 miles.

\section{
Highest Common Factor \& Least Common Multiple
}


\subsection{HCF}
What is the highest common factor of 72 and 45?

Idea: use prime factorization, look at what's common.

$72 = 2 \times 2 \times 2 \times 3 \times 3 $, $45 = 3 \times 3 \times 5$.

So HCF(72,15) = $ 3 \times 3 $ (the factors common to both).

\subsection{LCM}

What is the least common multiple of 24 and 15?

Idea:  get HCF (which is 3)
LCM is $\frac{24 \times 15}{HCF}$.

Keep things factored!

So LCM equals
$\frac{(2 \times 2 \times 2 \times 3) \times (3 \times 5)}{ 3} = 120$

What is the smallest even number that both 21 and 15 divide?

Prime factorization: $21 = 3 \times 7$, $15 = 3 \times 5$, so LCM is
$3 \times 5 \times 7 = 105$. It's odd, so the smallest even number
that both divide is $2 \times 105 = 210$.

\subsection{Number of divisors}

The number of distinct divisors of $2^4\times 5^7 \times 11^1 = (4+1)\times(7+1)\times(1+1)$

\section{Exponents}

$(2^3)^5 = 2^{3 \times 5} = 2^{15}$. (NOT $2^{8}$.)

$(2^3 \times 7^9)^5 = 2^{3 \times 5}  \times 7^{9 \times 5}= 2^{15} \times 7^{45}$

$3^5 \times 3^6 = 3^{11}$ (NOT $3^{5 \times 6}$.)

$\frac{2^5 \times 7^2}{2^3 \times 7^3} = \frac{2^{5-3}}{7^{3-1}}$.

\section{Miscellaneous/unclassified}
\begin{itemize}
\item leap years: February  2010 has how many days?
%\item division by a decimal: $\frac{12}{0.3}$
\item convert to decimal
\item place values (hundredths, tenths, units, tens, etc.)
\item concept of density
\item micro, milli, centi, deci, deca, hecta, kilo, mega, giga
\end{itemize}



\section{Problems to skip}
Which is biggest of some fractions?
\begin{itemize}
\item example: $\frac{7}{11}, \frac{6}{13}, \frac{5}{9}, \frac{2}{3}$
\item which is biggest: $\frac{11}{13}, \frac{12}{11}, \frac{14}{9}$?
\end{itemize}

% \section{Notes}

\section{Exponents}
\subsection*{Positive exponents}

Basic idea: $3^4$ is a short way to write $3 \times 3 \times 3 \times 3$.

General: write $a^b$.

Properties
\begin{itemize}
\item $2^{3+5} = 2^3 \times 2^5$. Reason---three copies of 2 and five copies of 2 makes eight copies of 2.
\item $(2 \times 3)^5 = 2^5 \times 3^5$. Reason---five copies of 2 and 3 is same as five copies of 2 and five copies of 3.
\item $(2^3)^4 = 2^{3 \times 4}$. Reason---4 copies of 3 copies of 2 is $4 \times 3 = 3 \times 4$.
\end{itemize}
Nonproperties: $2^3 \times 3^2$, $2^3 + 2^5$

General formulas: $a^{b + c} = a^b \times a^c$, $(a \times b)^c = a^c \times b^c$, $(a^b)^c = a^{b\times c}$

Memorize
\begin{itemize}
\item $2^1 = 2, 2^2 = 4, 2^3 = 8, 2^4 = 16, 2^5 = 32, 2^6 = 64, 2^7 = 128, 2^8 = 256, 2^9 = 512, 2^{10} = 1028$
\item $3^1 = 3, 3^2 = 9, 3^3 = 27, 3^4 = 81$
\item $4^1 = 4, 4^2 = 16, 4^3 = 64, 4^4 = 256$
\item $5^1 = 5, 5^2 = 25, 5^3 = 125, 5^4 = 625$
\item $6^1 = 5, 6^2 = 316, 6^3 = 218$
\end{itemize}

\subsection*{Zero exponent}

What should $2^0$ be? Would like $2^{0 + 1} = 2^0 \times 2^1$ (using formula from above). So $2^1 = 2^0 \times 2^1$.
Cancel out $2^1$ means $2^0$ must be $1$.

\subsection*{Negative exponents}

What should $2^{-1}$ be? Would like $2^{-1 + 1} = 2^{-1} \times 2^1$ (using formula from above). Since
$2^{-1 + 1} = 2^0 = 1$, it means $2^{-1} \times 2^1 = 1$, so $2^{-1} = \frac{1}{2^1}$.

General formula: $a^{-b} = \frac{1}{a^b}$---this is true even if $b$ is negative.

\subsection*{Fractional exponents}

What should $2^{\frac{1}{2}}$ be? Would like $2^\frac{1}{2} \times 2^\frac{1}{2} = 2^{\frac{1}{2} + \frac{1}{2}} = 2^1 = 2$.
So $2^\frac{1}{2} \times 2^\frac{1}{2} = 2$, meaning that $2^\frac{1}{2}$ is the square root of $2$.

Best way to understand something like $2^{\frac{3}{2}}$ is to use the $a^{b \times c} = {a^b}^c$ formula.
So $2^{\frac{3}{2}}$ is ${2^{\frac{1}{2}}}^3$, i.e., it's $\sqrt 2 \times \sqrt 2 \times \sqrt 2 = 2 \times \sqrt 2$.

Examples

$\frac{2^{-5/2}}{2^{-2}} = 2^{-5/2} \times {2^{- -2}} = 2^{-5/2} \times {2^{2}} = 2^{-5/2 + 2} = 2^{-1/2}$


\subsection{Distributivity property}

Basic idea: $2 \times ( 3 + 5 ) = 2 \times 3 + 2 \times 5$. (Think about rows and columns of rectangles, one is 2 high and 3 wide, other is 2 high and 5 wide.)

General formula: $a \times ( b + c) = a \times b + a \times c$.

\subsubsection*{Why replace $a \times ( b + c) = a \times b + a \times c$?}

$3 \times ( 10 + 231 ) - 3 \times 231$---by distributing we can cancel $3 \times 231$ like this:
$3 \times 10 + 3 \times 231 - 3 \times 231 = 3 \times 10$ (no complicated multiply by 231).

\subsubsection*{Why replace $a \times b + a \times c$ with $a \times ( b + c)$?}

$3 \times 64 + 3 \times 36 = 3 \times (64 + 36) = 3 \times 100$, makes for a simpler product.

Here's another reason you would want to go from $a \times b + a \times c$ to $a \times ( b + c)$.

Suppose you have to solve $x + 2 \times x + 4 \times x + 5 \times x = 144$.

This is pretty hard to solve by just looking at it. However, you can write $x + 2 \times x + 4 \times x + 5 \times x$
as $(1 + 2 + 4 + 5)\times x$ (since $x + 2 \times x + 4 \times x + 5 \times x = 1 \times x + 2 \times x + 4 \times x + 5 \times x$,
and distribution works when the $a$ is on the left or the right ($(b+c)\times a = b \times a + c \times a$).

Now the equation is $12 \times x = 144$, which is easy to solve (divide by $12$).

Another interesting application of distribution is simplifying $(1-x)\times(1 + x + x^2 + x^3)$. This becomes
$(1-x)\times 1 + (1-x)\times x + (1-x)\times x^2 + (1-x)\times x^3$ = $(1 - x) + (x - x^2)  + (x^2 - x^3) + (x^3 - x^4)$ = $1 - x^4$.

\end{document}
