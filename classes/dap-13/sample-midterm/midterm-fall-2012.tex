\documentclass[11pt]{exam}
\boxedpoints

\usepackage{xcolor}
\definecolor{very-light-gray}{gray}{0.97}
\definecolor{light-gray}{gray}{0.95}
\definecolor{dark-gray}{gray}{0.50}

\usepackage{listings}
\lstset{linewidth=1.0\textwidth}
\lstset{xleftmargin=0.20in}
\lstset{xrightmargin=0.20in}
\lstset{frame=tlrb}
\lstset{framexleftmargin=0mm}
\lstset{rulesepcolor=\color{dark-gray}}
\lstset{rulecolor=\color{dark-gray}}
\lstset{fillcolor=\color{very-light-gray}}
\lstset{numbers=left}
\lstset{stringstyle=\ttfamily}
\lstset{showstringspaces=false}
\lstset{numbersep=5pt}
\lstset{stepnumber=1}
\lstset{firstnumber=1}
\lstset{tabsize=4}
\lstset{keywordstyle=\color{black}\bfseries}
\lstset{breakatwhitespace=true}
\lstset{numberstyle=\tiny\color{black}}
\lstset{columns=fullflexible}
\lstset{breaklines=true}
\lstset{basicstyle=\footnotesize\ttfamily}
\lstset{backgroundcolor=\color{light-gray}}

\title{}

\author{}
\date{}
\addpoints
\pointsinmargin
\marginpointname{ marks}

\begin{document}

\maketitle

\large

{\bf Name and UTEID:}
\vfill

% \begin{center}
\begin{itemize}
% \item This exam has \numquestions\ questions for a total of \numpoints\ points. 
\item You are allowed 75 mins.
\item Closed everything.
\item Write your answers on the exam. 
\item Show your work and give explanations. 
\item No questions will be entertained---if you feel
a question is ambiguous or incomplete, make and state
reasonable assumptions. 
\end{itemize}

\vfill
\begin{center}
\pointtable[h][questions]
\end{center}

\vfill

\newpage

\begin{questions}

\question[16] {\bf Queue from two stacks}

Implement the {\tt Queue} class using the given {\tt Stack} class.
You must use the {\tt Stack} class only, specifically
the API that it exports. You cannot declare arrays
or linked lists in your code. Your code should not set
an upper bound on the number of elements in the queue.
Assume the {\tt EmptyException} class subclasses {\tt Eception}
and is already written.

Big hint: read the title of this problem

\begin{lstlisting}[language=Java]
class Stack {
  // various internal fields that are not shown
  // Public API - these are implemented as documented, and work correctly
  Stack() { ... } // constructor, returns a new, empty, Stack object
  Object pop() { ... } // returns most recently added element, throws
                         // EmptyException if the stack is empty
  void push(Object o) { ... } // add o to the stack
  boolean isEmpty() { ... } // returns true if the stack doesn not
                              // contain any elements
}

class Queue {
  // You have to implement this class. Specifically,
  // choose the right fields, and member functions (if any)

  // Public API - you have to implement these functions
  Queue() { ... } // constructor, returns a new, empty, Queue object
  Object dequeue() { ... } // returns least recently added element, throws
                             // EmptyException if the queue is empty
  void enqueue(Object o) { ... } // add o to the queue
  boolean isEmpty() { ... } // returns true if the queue doesn not
                              // contain any elements
}
\end{lstlisting}

%The queue should implement a no argument constructor,
%a nonstatic enque method (which takes an integer as an argument and adds it
%to the queue), and a nonstatic dequeue method (which takes no arguments,
%removes the integer which was enqueued earliest, and returns that integer).

%The stack library is just like in Lab~2, i.e., a constructor, a push method,
%and a pop method. 

%All you can do with a stack once its created is push, pop it.

%Write actual Java code, i.e., a class declaration, internal fields (if any), 
%the two methods (enque and dequeue), and the constructor.

\newpage
Overflow page.

\newpage
\question {\bf Java Puzzlers}

\begin{parts}
\part[4] Explain why the program below prints {\tt Za187}.

\begin{lstlisting}[language=Java]
class CharStringDiff {
  public static void main(String args[]) {
      System.out.print("Z" + "a");
      System.out.print('Z' + 'a');
  }
}
\end{lstlisting}
\vfill

\part[3] What does the following program print? Explain your answer.

\begin{lstlisting}[language=Java]
public class Animals {
  public static void main(String args[]) {
      final String pig = "length: 10";
      final String dog = "length: " + pig.length();
      System.out.println("Animals are equal: " + pig == dog);
  }
}
\end{lstlisting}
\vfill
\newpage

\part[3] Provide a declaration for {\tt i} that make this loop into an infinite loop.

\begin{lstlisting}[language=Java]
while (i != i) {
}
\end{lstlisting}
\vfill

\part[4]  What does this program print? Explain your answer.

\begin{lstlisting}[language=Java]
class Dog{
  public static void bark() {
      System.out.print("woof ");
  }
}

class Basenji extends Dog {
  public static void bark() { }
}

public class Bark {
  public static void main(String args[]) {
      Dog woofer = new Dog();
      Dog nipper = new Basenji();
      woofer.bark();
      nipper.bark();
  }
}
\end{lstlisting}
\vfill

\end{parts}

\newpage

\question {\bf Design a class}

In this problem you are to implement a simple date class. 
You should be able to 
represent any date from 1/1/1970  to 12/31/2500.
Internally, your representation should be the number
of days since 1/1/1970. 

A year is a leap year if it is divisible by 4 and not by 100
unless it is also divisible by 400. E.g., 2100 is not
a leap year, but 2000 is.

\begin{parts}

\part[7]
Write a constructor which takes as an argument three
ints (month, day, year), and returns a date object.
Throw a {\tt BadDateException} exception if the date is out of range.
(Assume the {\tt BadDateException} class already exists.)

\part[6]
Write a nonstatic method which increments the date object
by a specified number of days.
Throw a BadDateException exception if it result
in the date being out of range.

\part[8]
Override {\tt toString()} so that the date is printed
in the form {\tt 3/14/1997}.

\end{parts}

\newpage

Overflow page

\newpage


\question[16] {\bf Concurrency}

Consider an object $s$ which is read from and written to by many threads.
(For example, $s$ could be a cache similar to Lab~5.)
You need to ensure that no thread may access $s$
for reading or writing while another thread is writing to $s$.
(Two or more readers may access $s$ at the same time.)

One way to achieve this is by
protecting $s$ with a lock that ensures that no thread can access $s$ at the same
time as another writer.
However this solution is suboptimal because it is possible that a reader $R1$ has
locked $s$ and another reader $R2$ wants to access $s$.
Reader $R2$ does not have to wait until $R1$ is done reading; instead, $R2$
should start reading right away.

This motivates the first readers-writers problem:
protect $s$ with the added constraint that no reader is to be kept waiting if
$s$ is currently opened for reading.

Describe how you would solve the first readers-writers problem. Use actual Java
code, pseudo-code, figures and/or text, to describe your solution.

Hint: the number of readers is important.

Here is code for the suboptimal solution that is referred to in Paragraph 2 of
this problem.

\begin{lstlisting}[language=Java]
  // RW.s is a static field in the RW class.
  // It is the shared object s referred to in the problem text.
  // RW.lock is a static field in the RW class.
  // Its type is Object, and the only purpose it serves is to lock s.
  class Reader extends Thread {
    public void run() {
      while (true) {
        // Yuck - no other thread can read RW.s even though it's not being changed
        synchronized (RW.lock) {
          System.out.println(RW.s);
        }
        Task.doSomeThingElse();
      }
    }
  }

  class Writer extends Thread {
    public void run() {
      while (true) {
        synchronized (RW.lock) {
          RW.s = new Date().toString();
        }
        Task.doSomeThingElse();
      }
    }
  }

\end{lstlisting}

\newpage

Overflow page

\newpage

\question {\bf Anonymous classes}

Use the code snippet from Lab~3 below to 
answer the following: 
\begin{parts}

\part[2] What is an anonymous class?
\part[2] Why is an anonymous class useful?
\part[2] How can use of an anonymous class lead to wasted calls to object creation?
\end{parts}

\begin{lstlisting}[language=Java]
  interface StudentPredicate {
    boolean check( Student s );
  }

  Student [] filteredTestCase =  apply( testCase, new StudentPredicate() {
    public boolean check( Student s ) {
      return (s.GPA >= lowRange && s.GPA <= highRange );
    }
  });

  public static Student [] apply( Student [] input, StudentPredicate predicate ) {
    Student [] result = new Student[input.length];
    int i = 0;
    for ( Student s : input ) {
      if ( predicate.check( s ) ) {
        result[i++] = s;
      }
    }
    return result;
  }
\end{lstlisting}
\end{questions}


\end{document}
